\documentclass[]{article}
\usepackage{lmodern}
\usepackage{amssymb,amsmath}
\usepackage{ifxetex,ifluatex}
\usepackage{fixltx2e} % provides \textsubscript
\ifnum 0\ifxetex 1\fi\ifluatex 1\fi=0 % if pdftex
  \usepackage[T1]{fontenc}
  \usepackage[utf8]{inputenc}
\else % if luatex or xelatex
  \ifxetex
    \usepackage{mathspec}
  \else
    \usepackage{fontspec}
  \fi
  \defaultfontfeatures{Ligatures=TeX,Scale=MatchLowercase}
\fi
% use upquote if available, for straight quotes in verbatim environments
\IfFileExists{upquote.sty}{\usepackage{upquote}}{}
% use microtype if available
\IfFileExists{microtype.sty}{%
\usepackage{microtype}
\UseMicrotypeSet[protrusion]{basicmath} % disable protrusion for tt fonts
}{}
\usepackage[margin=1in]{geometry}
\usepackage{hyperref}
\hypersetup{unicode=true,
            pdftitle={Survival analysis week 2},
            pdfauthor={Petteri Mäntymaa},
            pdfborder={0 0 0},
            breaklinks=true}
\urlstyle{same}  % don't use monospace font for urls
\usepackage{color}
\usepackage{fancyvrb}
\newcommand{\VerbBar}{|}
\newcommand{\VERB}{\Verb[commandchars=\\\{\}]}
\DefineVerbatimEnvironment{Highlighting}{Verbatim}{commandchars=\\\{\}}
% Add ',fontsize=\small' for more characters per line
\usepackage{framed}
\definecolor{shadecolor}{RGB}{248,248,248}
\newenvironment{Shaded}{\begin{snugshade}}{\end{snugshade}}
\newcommand{\KeywordTok}[1]{\textcolor[rgb]{0.13,0.29,0.53}{\textbf{#1}}}
\newcommand{\DataTypeTok}[1]{\textcolor[rgb]{0.13,0.29,0.53}{#1}}
\newcommand{\DecValTok}[1]{\textcolor[rgb]{0.00,0.00,0.81}{#1}}
\newcommand{\BaseNTok}[1]{\textcolor[rgb]{0.00,0.00,0.81}{#1}}
\newcommand{\FloatTok}[1]{\textcolor[rgb]{0.00,0.00,0.81}{#1}}
\newcommand{\ConstantTok}[1]{\textcolor[rgb]{0.00,0.00,0.00}{#1}}
\newcommand{\CharTok}[1]{\textcolor[rgb]{0.31,0.60,0.02}{#1}}
\newcommand{\SpecialCharTok}[1]{\textcolor[rgb]{0.00,0.00,0.00}{#1}}
\newcommand{\StringTok}[1]{\textcolor[rgb]{0.31,0.60,0.02}{#1}}
\newcommand{\VerbatimStringTok}[1]{\textcolor[rgb]{0.31,0.60,0.02}{#1}}
\newcommand{\SpecialStringTok}[1]{\textcolor[rgb]{0.31,0.60,0.02}{#1}}
\newcommand{\ImportTok}[1]{#1}
\newcommand{\CommentTok}[1]{\textcolor[rgb]{0.56,0.35,0.01}{\textit{#1}}}
\newcommand{\DocumentationTok}[1]{\textcolor[rgb]{0.56,0.35,0.01}{\textbf{\textit{#1}}}}
\newcommand{\AnnotationTok}[1]{\textcolor[rgb]{0.56,0.35,0.01}{\textbf{\textit{#1}}}}
\newcommand{\CommentVarTok}[1]{\textcolor[rgb]{0.56,0.35,0.01}{\textbf{\textit{#1}}}}
\newcommand{\OtherTok}[1]{\textcolor[rgb]{0.56,0.35,0.01}{#1}}
\newcommand{\FunctionTok}[1]{\textcolor[rgb]{0.00,0.00,0.00}{#1}}
\newcommand{\VariableTok}[1]{\textcolor[rgb]{0.00,0.00,0.00}{#1}}
\newcommand{\ControlFlowTok}[1]{\textcolor[rgb]{0.13,0.29,0.53}{\textbf{#1}}}
\newcommand{\OperatorTok}[1]{\textcolor[rgb]{0.81,0.36,0.00}{\textbf{#1}}}
\newcommand{\BuiltInTok}[1]{#1}
\newcommand{\ExtensionTok}[1]{#1}
\newcommand{\PreprocessorTok}[1]{\textcolor[rgb]{0.56,0.35,0.01}{\textit{#1}}}
\newcommand{\AttributeTok}[1]{\textcolor[rgb]{0.77,0.63,0.00}{#1}}
\newcommand{\RegionMarkerTok}[1]{#1}
\newcommand{\InformationTok}[1]{\textcolor[rgb]{0.56,0.35,0.01}{\textbf{\textit{#1}}}}
\newcommand{\WarningTok}[1]{\textcolor[rgb]{0.56,0.35,0.01}{\textbf{\textit{#1}}}}
\newcommand{\AlertTok}[1]{\textcolor[rgb]{0.94,0.16,0.16}{#1}}
\newcommand{\ErrorTok}[1]{\textcolor[rgb]{0.64,0.00,0.00}{\textbf{#1}}}
\newcommand{\NormalTok}[1]{#1}
\usepackage{graphicx,grffile}
\makeatletter
\def\maxwidth{\ifdim\Gin@nat@width>\linewidth\linewidth\else\Gin@nat@width\fi}
\def\maxheight{\ifdim\Gin@nat@height>\textheight\textheight\else\Gin@nat@height\fi}
\makeatother
% Scale images if necessary, so that they will not overflow the page
% margins by default, and it is still possible to overwrite the defaults
% using explicit options in \includegraphics[width, height, ...]{}
\setkeys{Gin}{width=\maxwidth,height=\maxheight,keepaspectratio}
\IfFileExists{parskip.sty}{%
\usepackage{parskip}
}{% else
\setlength{\parindent}{0pt}
\setlength{\parskip}{6pt plus 2pt minus 1pt}
}
\setlength{\emergencystretch}{3em}  % prevent overfull lines
\providecommand{\tightlist}{%
  \setlength{\itemsep}{0pt}\setlength{\parskip}{0pt}}
\setcounter{secnumdepth}{0}
% Redefines (sub)paragraphs to behave more like sections
\ifx\paragraph\undefined\else
\let\oldparagraph\paragraph
\renewcommand{\paragraph}[1]{\oldparagraph{#1}\mbox{}}
\fi
\ifx\subparagraph\undefined\else
\let\oldsubparagraph\subparagraph
\renewcommand{\subparagraph}[1]{\oldsubparagraph{#1}\mbox{}}
\fi

%%% Use protect on footnotes to avoid problems with footnotes in titles
\let\rmarkdownfootnote\footnote%
\def\footnote{\protect\rmarkdownfootnote}

%%% Change title format to be more compact
\usepackage{titling}

% Create subtitle command for use in maketitle
\providecommand{\subtitle}[1]{
  \posttitle{
    \begin{center}\large#1\end{center}
    }
}

\setlength{\droptitle}{-2em}

  \title{Survival analysis week 2}
    \pretitle{\vspace{\droptitle}\centering\huge}
  \posttitle{\par}
    \author{Petteri Mäntymaa}
    \preauthor{\centering\large\emph}
  \postauthor{\par}
      \predate{\centering\large\emph}
  \postdate{\par}
    \date{March 20, 2019}


\begin{document}
\maketitle

\subsection{Exercise 2: Paramteric survival models and model
checking}\label{exercise-2-paramteric-survival-models-and-model-checking}

\subsubsection{1. Properties of exponential
distribution}\label{properties-of-exponential-distribution}

\paragraph{1.1.}\label{section}

Let \(T_{1},\dots, T_{n}\) be a random sample from a distribution with
survival function \(S(t) = exp\{−\lambda t\}\). Show that the
distribution of \(T = nmin(T1,\dots, Tn)\) is exponential with failure
rate \(\lambda\).

Note: you may prove that this result (in limit) holds even when \(S(t)\)
is such that \(S(t) = 1 − \lambda t + o(t)\) as \(t \rightarrow 0\)
where \(o(t)\) means \(\frac{o(t)}{t} \rightarrow 0\) as
\(t \rightarrow 0\).

\paragraph{1.2.}\label{section-1}

Show that the exponential distribution is the only continuous
distribution for which the mean residual lifetime \(r(t)\) is constant
for all \(t > 0\).

\paragraph{1.3.}\label{section-2}

Show that the nth moment of the exponential distribution with falire
rate \(\lambda\) is E(T n) = n! λn.

\paragraph{2. Fitting exponential and Weibull model to Veteran
data}\label{fitting-exponential-and-weibull-model-to-veteran-data}

Load veteran data from library(survival).

\paragraph{2.1.}\label{section-3}

Plot a histogram of the survival times corresponding to uncensored
observations \texttt{(veteran\$status\ ==\ 1)} as done in Exercise 1.

\begin{Shaded}
\begin{Highlighting}[]
\NormalTok{veteran }\OperatorTok
\StringTok{  }\KeywordTok{ggplot}\NormalTok{() }\OperatorTok{+}
\StringTok{  }\KeywordTok{aes}\NormalTok{(}\DataTypeTok{x =}\NormalTok{ time, }\DataTypeTok{fill =} \KeywordTok{factor}\NormalTok{(status)) }\OperatorTok{+}
\StringTok{  }\KeywordTok{geom_histogram}\NormalTok{(}\DataTypeTok{binwidth =} \DecValTok{40}\NormalTok{, }\DataTypeTok{color =} \StringTok{"black"}\NormalTok{) }\OperatorTok{+}
\StringTok{  }\KeywordTok{theme_minimal}\NormalTok{() }\OperatorTok{+}
\StringTok{  }\KeywordTok{labs}\NormalTok{(}\DataTypeTok{title =} \StringTok{"Survival time of individuals"}\NormalTok{, }\DataTypeTok{subtitle =} \StringTok{"Veterans' Administration Lung Cancer study"}\NormalTok{) }\OperatorTok{+}
\StringTok{  }\KeywordTok{xlab}\NormalTok{(}\StringTok{"Survival time (days)"}\NormalTok{) }\OperatorTok{+}
\StringTok{  }\KeywordTok{ylab}\NormalTok{(}\StringTok{"Frequency"}\NormalTok{)}
\end{Highlighting}
\end{Shaded}

\includegraphics{survival_week2_files/figure-latex/unnamed-chunk-1-1.pdf}

\begin{quote}
Minor addition to last weeks histogram; The colour indicates censoring
status, red for censored (survival time unknown) and blue for
non-censored (survival time known, i.e.~event occurs in the study
period).
\end{quote}

\paragraph{2.2.}\label{section-4}

Compare the Kaplan-Meier estimate of the survival function to (a)
Exponential distribution, and (b) Weibull distribution. Use graphical
procedure and interpret the results.

Hint: You can obtain the maximum likelihood estimates of the parameters
using \texttt{weibreg()} function of \texttt{eha} package. For example,
the estimate of the parameter \(\lambda\) in
\(S(t) = exp\{−\lambda t\}\) is obtained using

\begin{quote}
Vertaa jakaumia kuvassa,
\end{quote}

\begin{Shaded}
\begin{Highlighting}[]
\NormalTok{veteran.exp0 <-}\StringTok{ }\KeywordTok{weibreg}\NormalTok{(}\DataTypeTok{formula =} \KeywordTok{Surv}\NormalTok{(time, status) }\OperatorTok{~}\StringTok{ }\DecValTok{1}\NormalTok{, }\DataTypeTok{data=}\NormalTok{veteran, }\DataTypeTok{shape=}\DecValTok{1}\NormalTok{)}
\NormalTok{lambda0 <-}\StringTok{ }\KeywordTok{exp}\NormalTok{(}\OperatorTok{-}\NormalTok{veteran.exp0}\OperatorTok{$}\NormalTok{coeff[}\DecValTok{1}\NormalTok{])}

\KeywordTok{plot}\NormalTok{(veteran.exp0)}
\end{Highlighting}
\end{Shaded}

\includegraphics{survival_week2_files/figure-latex/unnamed-chunk-2-1.pdf}

The parameters of a Weibull distribution
\(S(t) = exp\{−(\frac{t}{b})^a\}\) are estimated by

\begin{Shaded}
\begin{Highlighting}[]
\NormalTok{veteran.weibull0 <-}\StringTok{ }\KeywordTok{weibreg}\NormalTok{(}\DataTypeTok{formula =} \KeywordTok{Surv}\NormalTok{(time, status) }\OperatorTok{~}\StringTok{ }\DecValTok{1}\NormalTok{, }\DataTypeTok{data=}\NormalTok{veteran)}
\NormalTok{b <-}\StringTok{ }\KeywordTok{exp}\NormalTok{(veteran.weibull0}\OperatorTok{$}\NormalTok{coeff[}\DecValTok{1}\NormalTok{])}
\NormalTok{a <-}\StringTok{ }\KeywordTok{exp}\NormalTok{(veteran.weibull0}\OperatorTok{$}\NormalTok{coeff[}\DecValTok{2}\NormalTok{])}
\end{Highlighting}
\end{Shaded}

\paragraph{Model choice}\label{model-choice}

\paragraph{2.3.}\label{section-5}

Compare the above two models with the likelihood ratio test. Interpret
the result. Hint: You can extract the log-likelihood values from the
output objects of function weibreg. Use the pchisq function to calculate
the p-value (tail probability). Alternative: You can calculate the
likelihood ratio by using the anova command on the output objects from
the two regression models using survreg.

\paragraph{3. Simulation}\label{simulation}

\paragraph{3.1.}\label{section-6}

Generate 100 random numbers from exponential distribution with mean 0.01
and store it in T. Before you do this, look at the help(rexp) and the
parameter which it accepts.

\paragraph{3.1.1.}\label{section-7}

Plot the empirical distribution function.

\paragraph{3.1.2.}\label{section-8}

Esimate the rate (stored under obsrate) from the simulated data and
overlay the plot of the distribution function 1 − exp(−obsrate ∗ t).

\paragraph{3.1.3.}\label{section-9}

Overlay the plot of the true exponential distribution function.

\paragraph{3.1.4.}\label{section-10}

Explore the possibilities for different kinds of line and point plots.
Vary the plot symbol, line type, line width, and colour. Also, try to
give legend in the above graph.


\end{document}
